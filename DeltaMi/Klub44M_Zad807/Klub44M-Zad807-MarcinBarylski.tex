% !TEX TS-program = pdflatex
% !TEX encoding = UTF-8 Unicode

% This is a simple template for a LaTeX document using the "article" class.
% See "book", "report", "letter" for other types of document.

\documentclass[10pt,onecolumn]{article}

\usepackage[utf8]{inputenc} % set input encoding (not needed with XeLaTeX)
\usepackage{graphicx}
\usepackage{listings} 
\usepackage{caption} 
\usepackage{xcolor,colortbl}
\usepackage{mathtools}
\usepackage{wrapfig}
\usepackage{tabularx}
\usepackage{pdflscape}

\graphicspath{ {figures/} }

%%% Examples of Article customizations
% These packages are optional, depending whether you want the features they provide.
% See the LaTeX Companion or other references for full information.

%%% PAGE DIMENSIONS
\usepackage{geometry} % to change the page dimensions
\geometry{a4paper} % or letterpaper (US) or a5paper or....
\geometry{margin=1in} % for example, change the margins to 2 inches all round
% \geometry{landscape} % set up the page for landscape
%   read geometry.pdf for detailed page layout information
% \usepackage[parfill]{parskip} % Activate to begin paragraphs with an empty line rather than an indent

%%% PACKAGES
\usepackage{polski}
\usepackage{booktabs} % for much better looking tables
\usepackage{array} % for better arrays (eg matrices) in maths
\usepackage{paralist} % very flexible & customisable lists (eg. enumerate/itemize, etc.)
\usepackage{verbatim} % adds environment for commenting out blocks of text & for better verbatim
\usepackage{subfig} % make it possible to include more than one captioned figure/table in a single float
\usepackage{indentfirst}
\usepackage{amsfonts}
\usepackage{amssymb}
\usepackage{amsthm}

% These packages are all incorporated in the memoir class to one degree or another...

%%% HEADERS & FOOTERS
\usepackage{fancyhdr} % This should be set AFTER setting up the page geometry
\pagestyle{fancy} % options: empty , plain , fancy
\renewcommand{\headrulewidth}{0pt} % customise the layout...
\lhead{}\chead{}\rhead{}
\lfoot{}\cfoot{\thepage}\rfoot{}

%%% SECTION TITLE APPEARANCE
\usepackage{sectsty}
\allsectionsfont{\sffamily\mdseries\upshape} % (See the fntguide.pdf for font help)
% (This matches ConTeXt defaults)

%%% ToC (table of contents) APPEARANCE
\usepackage[nottoc,notlof,notlot]{tocbibind} % Put the bibliography in the ToC
\usepackage[titles,subfigure]{tocloft} % Alter the style of the Table of Contents
\renewcommand{\cftsecfont}{\rmfamily\mdseries\upshape}
\renewcommand{\cftsecpagefont}{\rmfamily\mdseries\upshape} % No bold!
\setlength{\parindent}{0.5cm} 

%%% END Article customizations

%%% The "real" document content comes below...

\definecolor{Gray}{gray}{0.85}
\definecolor{LightCyan}{rgb}{0.88,1,1}
\newcolumntype{a}{>{\columncolor{Gray}}c}
\newcolumntype{b}{>{\columncolor{white}}c}
\newcolumntype{L}[1]{>{\raggedright\arraybackslash}p{#1}}
\newcolumntype{C}[1]{>{\centering\arraybackslash}p{#1}}
\newcolumntype{R}[1]{>{\raggedleft\arraybackslash}p{#1}}

\newcommand{\subsubfloat}[2]{%
  \begin{tabular}{@{}c@{}}#1\\#2\end{tabular}%
}

\title{Zadanie 807 z Klubu 44M}
\author{Marcin Barylski}
\date{\small{28 październik, 2020 \\ Gdańsk}}

\begin{document}
\maketitle

\section{Treść zadania}

Dane są liczby $A, B \textgreater 0$; $AB \textless 1$. Funkcje $f : \mathbb{R} \rightarrow \mathbb{R},  g : \mathbb{R} \rightarrow \mathbb{R}$  spełniają dla wszystkich $x,y \in \mathbb{R}$  warunki: \par
$ \lvert f(x)-f(y) \rvert \leq A \lvert x-y \rvert, \lvert g(x)-g(y) \rvert \leq B \lvert x-y \rvert $ \par
przy czym  $f$ jest różnowartościowym odwzorowaniem zbioru $\mathbb{R}$  na cały zbiór $\mathbb{R}$ ; ma więc funkcję odwrotną $h : \mathbb{R} \rightarrow \mathbb{R} ( f(h(x)) = h( f (x)) = x)$. \par Udowodnić, że funkcja $g + h$  też jest różnowartościowym odwzorowaniem zbioru $\mathbb{R}$  na cały zbiór $\mathbb{R}$. 

\section{Rozwiązanie}

Funkcja odwrotna do funkcji oryginalnej ma oś symetrii $k(x) = x$. Zatem jeśli:

\begin{enumerate}[(1)]
\item $ \lvert f(x)-f(y) \rvert \leq A \lvert x-y \rvert$
\end{enumerate}

to funkcja odwrotna $h$ spełnia:

\begin{enumerate}[(2)]
\item $ \lvert h(x)-h(y) \rvert \geq \frac{1}{A} \lvert x-y \rvert$
\end{enumerate}

Z założenia zadania:

\begin{enumerate}[(3)]
\item $AB \textless 1$
\end{enumerate}

Dzieląc obustronnie (3) przez $A$ ($A$ jest dodatnie, zgodnie z założeniami, a więc nie ma zmiany znaku nierówności):

\begin{enumerate}[(4)]
\item $B \textless \frac{1}{A}$
\end{enumerate}

Łącząc założenie (2) z (4) i założeniem o $g$:

\begin{enumerate}[(5)]
\item $ \lvert h(x)-h(y) \rvert \geq \frac{1}{A} \lvert x-y \rvert \textgreater B \lvert x-y \rvert \geq  \lvert g(x)-g(y) \rvert$
\end{enumerate}

Funkcja $f$ jest różnowartościowym odwzorowaniem zbioru $\mathbb{R}$  na cały zbiór $\mathbb{R}$ z własnością (1), tak więc jest albo rosnąca, albo malejąca. Funkcja odwrotna zachowuje monotoniczność funkcji oryginalnej - $h$ jest nadal odpowiednio albo rosnąca, albo malejąca. \par Z (5) wynika, że przyrost wartości funkcji $h$ dla danej pary argumentów $x, y$ jest zawsze większy od przyrostu wartości funkcji $g$, czyli $g$ nie może zmienić monotoniczności $h$ w $h+g$, nawet jeśli jest przeciwnej monotoniczności do $h$. \par Dziedziną $h+g$ pozostaje zbiór $\mathbb{R}$, zaś przeciwdziedziną - cały zbiór $\mathbb{R}$. Monotoniczność $h+g$ daje różnowartościowość tego odwzorowania.

\textbf{Wniosek końcowy}: $g + h$  jest różnowartościowym odwzorowaniem zbioru $\mathbb{R}$  na cały zbiór $\mathbb{R}$.  \\ \\ \\

\end{document}
